\documentclass[10pt]{beamer}

\usepackage[T2A]{fontenc}
\usepackage[utf8]{inputenc}
\usepackage[russian,english]{babel}
\usepackage{subfig}
\usefonttheme[onlymath]{serif}

\usetheme[progressbar=frametitle]{metropolis}
\usepackage{appendixnumberbeamer}

\usepackage{booktabs}
\usepackage[scale=2]{ccicons}

\usepackage{pgfplots}
\usepgfplotslibrary{dateplot}

\usepackage{xspace}
\newcommand{\themename}{\textbf{\textsc{metropolis}}\xspace}
\newcommand{\TODO}[1]{\textbf{\textcolor{red}{TODO: #1}}}

\date{}
\author{Екатерина Тузова}


\title{Лекция 3}
\subtitle{Кластеризация}

\begin{document}

\maketitle

\section{Разбор летучки}

\section{Мотивирующий пример}

{\foot{\href{https://www.kaggle.com/abcsds/pokemon}{Pokemon with stats (https://www.kaggle.com/abcsds/pokemon)}}
\begin{frame}{Мотивирующий пример}
	\begin{figure}
	    \centering
	    \subfloat{{\includegraphics[width=2cm]{../lecture2/images/Bulbasaur} }}
	    \qquad
	    \subfloat{{\includegraphics[width=2cm]{../lecture2/images/Mewtwo} }}
    	    \qquad
    	    \subfloat{{\includegraphics[width=2cm]{../lecture2/images/Volcanion} }}
    	    \qquad
    	    \subfloat{{\includegraphics[width=2cm]{../lecture2/images/Ekans} }}
    	    \qquad
    	    \subfloat{{\includegraphics[width=2cm]{../lecture2/images/Nidorina} }}
	    \qquad 
    	    \subfloat{{\includegraphics[width=2cm]{../lecture2/images/Rattata} }}
	    \qquad
    	    \subfloat{{\includegraphics[width=2cm]{../lecture2/images/Sandshrew} }}
	    \qquad
    	    \subfloat{{\includegraphics[width=2cm]{../lecture2/images/Articuno} }}    	        	    
	\end{figure}
\end{frame}
}

\begin{frame}{Датасет}
    \centering
	\includegraphics[width=\textwidth]{../lecture2/images/pokemons}
\end{frame}

\begin{frame}{Постановка задачи кластеризации}
  Кластеризация -- задача разделения объектов одной природы на несколько групп так, чтобы объекты в одной группе обладали одним и тем же свойством.\\
  \bigbreak
  \pause
  Кластеризация -- это обучение без учителя.

\end{frame}

\begin{frame}{Постановка задачи кластеризации}
	$X$ -- пространство объектов\\
	$\rho: X \times X \rightarrow [0, \infty)$ -- функция расстояния между объектами\\
	\bigbreak
	\alert{Найти}:\\
	$Y$ -- множество кластеров \\
	$a: X \rightarrow Y$ -- алгоритм кластеризации
\end{frame}

\section{Гипотеза компактности}

\section{Какие функции расстояния мы знаем?}

\section{Степени свободы в постановке задачи}

\begin{frame}{Степени свободы в постановке задачи}
	\begin{itemize} [<+- | alert@+>]
		\item[--] Критерий качества кластеризации
		\item[--] Число кластеров неизвестно заранее
		\item[--] Результат кластеризации существенно зависит от метрики
	\end{itemize}
\end{frame}

\section{Цели кластеризации}

\begin{frame}{Цели кластеризации}
	\begin{itemize} [<+- | alert@+>]
		\item[--] Сократить объём хранимых данных
		\item[--] Выделить нетипичные объекты
		\item[--] Упростить дальнейшую обработку данных
		\item[--] Построить иерархию множества объектов				
	\end{itemize}
\end{frame}

\section{Какие бывают кластеры?}

\begin{frame}{Сгущения}
	\begin{center}
    \includegraphics[height=0.6 \textheight, keepaspectratio = true]{images/cluster1}  
	\end{center}
\end{frame}

\begin{frame}{С центром}
	\begin{center}
	  \includegraphics[height=0.6 \textheight, keepaspectratio = true]{images/cluster3}  
	\end{center}
\end{frame}

\begin{frame}{С перемычками}
	\begin{center}
	  \includegraphics[height=0.6 \textheight, keepaspectratio = true]{images/cluster4}  
	\end{center}
\end{frame}

\begin{frame}{Ленты}
	\begin{center}
	  \includegraphics[height=0.6 \textheight, keepaspectratio = true]{images/cluster2}  
	\end{center}
\end{frame}

\begin{frame}{На фоне}
	\begin{center}
	  \includegraphics[height=0.6 \textheight, keepaspectratio = true]{images/cluster5}  
	\end{center}
\end{frame}

\begin{frame}{Перекрывающиеся}
	\begin{center}
	  \includegraphics[height=0.6 \textheight, keepaspectratio = true]{images/cluster6}  
	\end{center}
\end{frame}

\section{Чувствительность к нормировке признаков}

\begin{frame}{Чувствительность к нормировке}
	\begin{center}
	  \includegraphics[height=0.8 \textheight, keepaspectratio = true]{images/weight_height1}  
	\end{center}
\end{frame}

\begin{frame}{Чувствительность к нормировке}
	\begin{center}
	  \includegraphics[height=0.8 \textheight, keepaspectratio = true]{images/weight_height2}  
	\end{center}
\end{frame}

\begin{frame}{Чувствительность к нормировке}
	\begin{center}
	  \includegraphics[height=0.8 \textheight, keepaspectratio = true]{images/weight_height3}  
	\end{center}
\end{frame}

\begin{frame}{Чувствительность к нормировке}
	\begin{center}
	  \includegraphics[height=0.8 \textheight, keepaspectratio = true]{images/weight_height4}  
	\end{center}
\end{frame}

%\section{Оценка качества кластеризации}
%
%\begin{frame}{Оценка качества кластеризации}
%  \alert{Идея}: Минимизировать среднее внутрикластерное расстояние и при этом максимизировать среднее межкластерное расстояние.
%	\bigbreak
%	\pause
%	$${\frac{\sum\limits_{a(x_i) = a(x_j)} \rho(x_i, x_j)}{\sum\limits_{a(x_i) = a(x_j)} 1} \rightarrow \min}$$
%	\bigbreak
%	\pause
%	$${\frac{\sum\limits_{a(x_i) \neq a(x_j)} \rho(x_i, x_j)}{\sum\limits_{a(x_i) \neq a(x_j)} 1} \rightarrow \max}$$
%\end{frame}

\begin{frame}{Методы кластеризации}
	\begin{enumerate} [-]
    \item Статистические 				
		\item Графовые 
		\item Иерархические
	\end{enumerate}
\end{frame}

\section{Статистические алгоритмы}

%\begin{frame}{Алгоритм FOREL}
%  \alert{Идея}:\\
%	\begin{itemize}
%		\item[--] Выделить все точки выборки $x_i$, попадающие внутрь сферы $\rho(x_i, x_0) \leq R$
%		\item[--] Перенести $x_0$ в центр тяжести выделенных точек
%		\item[--] Повторять пока $x_0$ не стабилизируется
%	\end{itemize}
%\end{frame}
%
%\begin{frame}{Алгоритм FOREL}
%	\begin{algorithmic}[1]
%        \Function{forel}{$X$, $R$}
%            \State ${U \gets X}$, $C \gets 0$
%            \While{${U \neq 0}$}                
%                \State выбрать случайную точку $x_0$
%                \MRepeat [пока $x_0$ не стабилизируется] 
%                		\State ${c = \left\{ x \in X \vert \rho(x, x_0) < R \right\}}$
%                		\State $x_0 = \frac{1}{\vert c \vert} \sum\limits_{x_i \in c} x_i$
%                	\EndRepeat
%    		        \State ${U = U \setminus c}$, ${C = C \cup \left\{ c \right\}}$
%            \EndWhile            
%        \EndFunction
%    \end{algorithmic}
%\end{frame}
%
%\begin{frame}{Алгоритм FOREL}
%	\begin{itemize} [<+- | alert@+>]
%		\item[+] Наглядность
%		\item[+] Сходимость
%		\bigbreak
%		\item[--] Зависимость от выбора $x_0$
%		\item[--] Плохо работает, если изначальная выборка плохо делится на кластеры
%	\end{itemize}
%\end{frame}

% можно какую-нибудь иллюстрацию придумать

{\foot{k-means}
\begin{frame}{Метод $k$-средних}
	\alert{Идея}: Мы можем искать центры кластеров путем усреднения вектора признаков объектов.
	\pause
	\bigbreak
	$${\sum\limits_{i = 1}^l \Vert x_i -\mu_i \Vert^2 \rightarrow \min}$$\\
	\bigbreak
	$\mu_i$ -- ближайший к $x_i$ центр кластера
\end{frame}
}

\begin{frame}{Метод $k$-средних}
	\begin{algorithmic}[1]
        \Function{kmeans}{$k$}
            \State Инициализировать $\mu_i$, $i = 1 \dots k$
            \MRepeat [пока $\mu_c$ не перестанет меняться] 
                \State $c_i = \arg\min\limits_{c \in 1\dots k} \rho(x_i, \mu_c)$ \hspace{5mm} $i = 1,\dots, l$\\
                
                \State ${\mu_{c} = \frac{\sum\limits_{j=1,\dots, n} [c_i = c] x_i^j}{\sum\limits_{c_i = c} n} }$ \hspace{10mm} $c \in 1 \dots k$
             \EndRepeat
        \EndFunction
  \end{algorithmic}    
	\bigbreak
	$\mu_{c}$ -- новое положение центров кластеров\\
	$c_i$ -- принадлежность $x_i$ к кластеру\\
	$\rho(x_i, \mu_c)$ -- расстояние от $x_i$ до центра кластера $\mu_c$
\end{frame}

\begin{frame}{Метод $k$-средних}
	\begin{center}
	  \includegraphics[width= \textwidth, keepaspectratio = true]{images/kmeans1}   
	\end{center}
\end{frame}

\begin{frame}{Метод $k$-средних}
	\begin{center}
	  \includegraphics[width= \textwidth, keepaspectratio = true]{images/kmeans2}   
	\end{center}
\end{frame}

\begin{frame}{Метод $k$-средних}
	\begin{center}
	  \includegraphics[width= \textwidth, keepaspectratio = true]{images/kmeans3}   
	\end{center}
\end{frame}

\begin{frame}{Метод $k$-средних}
	\begin{center}
	  \includegraphics[width= \textwidth, keepaspectratio = true]{images/kmeans4}   
	\end{center}
\end{frame}

\begin{frame}{Особенности метода $k$-средних}
	\begin{itemize}
		\item[--] Чувствительность к начальному выбору $\mu_c$
		\item[--] Необходимость задавать $k$
	\end{itemize}
\end{frame}

\begin{frame}{Чувствительность к начальному выбору $\mu_c$}
	\begin{center}
	  \includegraphics[width= \textwidth, keepaspectratio = true]{images/localmin1}  
	\end{center}
\end{frame}

\begin{frame}{Чувствительность к начальному выбору $\mu_c$}
	\begin{center}
	  \includegraphics[width= \textwidth, keepaspectratio = true]{images/localmin2}  
	\end{center}
\end{frame}

\begin{frame}{Чувствительность к начальному выбору $\mu_c$}
	\begin{center}
	  \includegraphics[width= \textwidth, keepaspectratio = true]{images/localmin3}  
	\end{center}
\end{frame}

\begin{frame}{Чувствительность к начальному выбору $\mu_c$}
	\begin{center}
	  \includegraphics[width= \textwidth, keepaspectratio = true]{images/localmin4}  
	\end{center}
\end{frame}

\begin{frame}{Необходимость задавать $k$}
	\begin{center}
	  \includegraphics[width= \textwidth, keepaspectratio = true]{images/k_means_k}  
	\end{center}
\end{frame}

\begin{frame}{Необходимость задавать $k$}
	\begin{center}
	  \includegraphics[width= \textwidth, keepaspectratio = true]{images/k_means_k4}  
	\end{center}
\end{frame}

\begin{frame}{Необходимость задавать $k$}
	\begin{center}
	  \includegraphics[width= \textwidth, keepaspectratio = true]{images/k_means_k7}  
	\end{center}
\end{frame}

\section{Устранение недостатков}

\begin{frame}{Инициализация $\mu_c$}
  \begin{itemize} [<+->]
    \item Случайным образом
    \item Взять наиболее удалённые объекты выборки
    \item Несколько случайных кластеризаций
    \item Использование k-means++
  \end{itemize}
\end{frame}

\begin{frame}{k-means++}
	\alert{Идея}:\\
	\begin{enumerate}
		\item Выбрать первый центроид случайным образом
		\item Для каждой точки найти значение квадрата расстояния до ближайшего центроида.
		\item Выбрать из этих точек следующий центроид так, чтобы вероятность выбора точки была пропорциональна вычисленному для неё квадрату расстояния
	\end{enumerate}
\end{frame}

{\foot{\href{https://www.cs.cmu.edu/~dpelleg/download/xmeans.pdf}{https://www.cs.cmu.edu/~dpelleg/download/xmeans.pdf}}
\begin{frame}{X-means}
	\alert{Идея}:\\
	\begin{enumerate}
		\item Получать на вход не k, а диапазон, в котором может находиться k.
		\item Запустить k-means на самом маленьком значении из диапазона.
		\item Разбить пополам полученные кластеры и проверить, не улучшилась ли кластеризация.
	\end{enumerate}
\end{frame}
}

%\begin{frame}{X-means}
%\TODO{картинка}
%	\begin{figure}[htbp]
%	  \includegraphics[height=140pt, keepaspectratio = true]{images/x-means}  
%	    \includegraphics[height=140pt, keepaspectratio = true]{images/x-means-1}
%	\end{figure}
%https://www.cs.cmu.edu/~dpelleg/download/xmeans.pdf
%\end{frame}

% Слишком сложно?
%\begin{frame}{X-means}
%  \begin{center}
%    Как проверить, что кластеризация улучшилась?
%  \end{center}
%\end{frame}

%\begin{frame}{Байесовский информационный критерий}
%	$BIC_j = L_j(X)  + \frac{d}{2} \log(l)$\\
%	\bigbreak
%	$L_j$ -- логарифмическая функция правдоподобия для $j$-й модели \\
%	$d$ -- длина вектора параметров\\
%	$l$ -- количество объектов в выборке\\
%\end{frame}

\section{Интересные результаты}

{\foot{\href{https://www.reddit.com/r/pokemonconspiracies/comments/2uu7lc/arcanine_was_meant_to_be_a_legendary/
}{Pokemon Conspiracies}}
\begin{frame}{Интересные результаты}
	\begin{center}
	  	\includegraphics[keepaspectratio = true]{images/Arcanine}  \\
	  	Arcanine
	\end{center}
\end{frame}

\begin{frame}{Интересные результаты}
	\begin{center}
	  \includegraphics[width= \textwidth, height=0.8 \textheight, keepaspectratio = true]{images/legendary_arcanine}  
	\end{center}
\end{frame}
}

\section{Когда k-means работает плохо}

\begin{frame}{"Не сферические данные"}
	\begin{center}
	  \includegraphics[width=\textwidth, keepaspectratio = true]{images/non_spherical-1}  
	\end{center}
\end{frame}	

\begin{frame}{"Не сферические данные"}
	\begin{center}
	  \includegraphics[width=\textwidth, keepaspectratio = true]{images/non_spherical-2}  
	\end{center}
\end{frame}	

\begin{frame}{"Не сферические данные"}
	\begin{center}
	  \includegraphics[width=\textwidth, keepaspectratio = true]{images/non_spherical-3}  
	\end{center}
\end{frame}	

\begin{frame}{Разноразмерные кластеры}
	\begin{center}
	  \includegraphics[width=\textwidth, height=0.8 \textheight, keepaspectratio = true]{images/different_sizes-1}  
	\end{center}
\end{frame}

\begin{frame}{Разноразмерные кластеры}
	\begin{center}
	  \includegraphics[width=\textwidth, height=0.8 \textheight, keepaspectratio = true]{images/different_sizes-2}  
	\end{center}
\end{frame}

{\foot{Affinity Propagation}
\begin{frame}{Метод распространения близости}
  \alert{Идея}: Позволим объектам обновлять информацию друг о друге, для того, чтобы выбрать центр кластера.\\
  \pause
  \bigbreak
  $s(i, k)$ -- "похожесть" объекта $x_i$ на $x_k$, $s(k, k) < 0$.\\
  $r(i, k)$ -- "ответственность", $x_i$ решает насколько $x_k$ подходит для того, чтобы быть центром кластера.\\
  $a(i,k)$ -- "доступность", $x_k$ решает насколько подходит для того, чтобы быть центром кластера $x_i$.\\
  \bigbreak
\end{frame}
}

\begin{frame}{Метод распространения близости}
	\begin{algorithmic}[1]
        \Function{affinity\_propagation}{$S$}
            \State $R \gets 0$, $A \gets 0$            
            \MRepeat [пока $c_i$ не перестанет меняться] 
                \State $r(i, k) = s(i,k) - \max\limits_{j \neq k} (a(i,j) + s(i,j))$
                \State $a(k,k) = \sum\limits_{j \neq k} \max(0, r(j,k))$
                \State $i \neq k: a(i,k) = \min(0, r(k,k) + \sum\limits_{j \neq k} \max(0, r(j,k)))$
                \State $c_i = \arg\max\limits_{k} (a(i,k) + r(i,k))$                
             \EndRepeat
        \EndFunction
  \end{algorithmic}    
\end{frame}

\begin{frame}{Метод распространения близости}
  \begin{enumerate}[<+- | alert@+>]
    \item[+] Не нужно задавать количество кластеров
    \item[+] "Выбросы" выделяются в отдельные кластеры
    \bigbreak
    \item[--] Долгое время работы
    \item[--] Часто нуждается в постобработке
  \end{enumerate}
\end{frame}

\section{Графовые алгоритмы}

\begin{frame}{Графовые алгоритмы}
  \begin{center}
    Какие есть две очевидные идеи?
  \end{center}    
\end{frame}

\begin{frame}{Графовые алгоритмы}
	Идеи:\\
	\begin{enumerate}
		\item Выделение связных компонент
		\item Минимальное покрывающее дерево
	\end{enumerate}
\end{frame}

\begin{frame}{Выделение связных компонент}
	\begin{enumerate}
		\item Рисуем полный граф с весами, равными расстоянию между объектами
		\item Выбираем лимит расстояния $r$ и выкидываем все ребра\\ длиннее $r$
		\item Компоненты связности полученного графа -- наши кластеры
	\end{enumerate}
\end{frame}

\begin{frame}{Выделение связных компонент}
  \begin{center}
    Как искать \alert{компоненты связности}?
  \end{center}
\end{frame}

\begin{frame}{Минимальное покрывающее дерево}
	Минимальное остовное дерево -- дерево, содержащее все вершины графа и имеющее минимальный суммарный вес ребер.\\
\end{frame}

\begin{frame}{Минимальное покрывающее дерево}
  Как использовать минимальное остовное дерево для разбиения на кластеры?
\end{frame}

\begin{frame}{Минимальное покрывающее дерево}
	Строим минимальное остовное дерево, а потом выкидываем из него ребра максимального веса.\\
	\bigbreak
	Сколько ребер выбросим -- столько кластеров получим.
\end{frame}

\section{Иерархическая кластеризация}

\begin{frame}{Агломеративный алгоритм Ланса-Уильямса}
	\alert{Идея}:\\
	\begin{enumerate}
		\item Считаем каждую точку кластером. 
		\item Затем объединяем ближайшие точки в новый кластер. 
		\item Повторяем.
	\end{enumerate}
\end{frame}


\begin{frame}{Алгоритм Ланса-Уильямса}
	\begin{algorithmic}[1]
        \Function{lance-williams}{$X^l$}
            \State ${C_1 = \left\{ \left\{ x_1\right\}, \left\{x_2 \right\}, \dots, \left\{x_l \right\} \right\}}$
            \For {${i=2, \dots, l }$}
            		\State ${(U, V) = \arg\min\limits_{U \neq V} \rho(U, V)}$
            		\State $W = U \cup V$
            		\State ${C_i = C_{i-1} \cup \left\{ W \right\}\setminus \left\{U, V \right\} }$
            		\For {\textbf{each} ${S \in C_i}$}
            			\State вычислить $\rho(W, S)$
            		\EndFor
            	\EndFor
        \EndFunction
    \end{algorithmic}   
\end{frame}
	
\begin{frame}{Алгоритм Ланса-Уильямса}
  \begin{center}
    Чего-то не хватает?  
  \end{center}
\end{frame}

\begin{frame}\frametitle{Формула Ланса-Уильямса}
	\begin{minipage}[t]{0.45\linewidth}
    \vspace{15mm}
 		${ W = \left\{ U \cup V \right\} }$\\	
		\bigbreak		
		\alert{Знаем}:\\
		${\rho(U, S), \rho(V, S), \rho(U, V)}$
		\bigbreak				
		Расстояние $\rho(W, S)$?\\	
	\end{minipage}%
	\begin{minipage}[t]{0.55\linewidth}
	  \begin{figure}[htbp]
	    \includegraphics[height=150pt, keepaspectratio = true]{images/lans-formula}  
	  \end{figure}
  \end{minipage}%
\end{frame}

\begin{frame}{Формула Ланса-Уильямса}
	${ W = \left\{ U \cup V \right\} }$\\
	\bigbreak
	${\rho(U \cup V, S) = \alpha_U \rho(U, S) + \alpha_V \rho(V, S) + }$ \\
	\hspace{30mm} ${ + \beta \rho(U, V) + \gamma \vert \rho(U, S) - \rho(V, S)\vert}$\\
	\bigbreak
	${\alpha_U, \alpha_V, \beta, \gamma}$ -- числовые параметры
\end{frame}

\begin{frame}{Параметры}
	Значения параметров
	${\alpha_U, \alpha_V, \beta, \gamma}$ ?
\end{frame}

\begin{frame}{Расстояние ближнего соседа}
	\begin{figure}[htbp]
	  \includegraphics[height=150pt, keepaspectratio = true]{images/lans1}  
	\end{figure}
	\pause
	${\alpha_U = \alpha_V = \frac{1}{2}}$ \\${\beta = 0}$ \\${\gamma = -\frac{1}{2}}$
\end{frame}

\begin{frame}{Расстояние дальнего соседа}
	\begin{figure}[htbp]
	  \includegraphics[height=150pt, keepaspectratio = true]{images/lans2}  
	\end{figure}
	\pause
	${\alpha_U = \alpha_V = \frac{1}{2}}$ \\${\beta = 0}$ \\${\gamma = \frac{1}{2}}$
\end{frame}

\begin{frame}{Групповое среднее}
	\begin{figure}[htbp]
	  \includegraphics[height=150pt, keepaspectratio = true]{images/lans3}  
	\end{figure}
	\pause
	${\alpha_U = \frac{\vert U \vert}{\vert W \vert}}$\\${\alpha_V = \frac{\vert V \vert}{\vert W \vert}}$ \\${\beta = \gamma = 0}$
\end{frame}

\begin{frame}{Расстояние Уорда}
	\begin{figure}[htbp]
	  \includegraphics[height=150pt, keepaspectratio = true]{images/lans4}  
	\end{figure}

  ${\alpha_U = \frac{\vert S \vert + \vert U \vert}{\vert S \vert + \vert W \vert}}$\\${\alpha_V = \frac{\vert S \vert + \vert V \vert}{\vert S \vert + \vert W \vert}}$ \\${\beta = \frac{ -\vert S \vert}{\vert S \vert + \vert W \vert}}$ \\${\gamma = 0}$
\end{frame}

\section{Визуализация кластеров}

{\foot{Treemap}
\begin{frame}{Диаграмма вложения}
  \begin{center}
    \includegraphics[height=0.8 \textheight, keepaspectratio = true]{images/diagram}    
  \end{center}
\end{frame}
}

\begin{frame}{Дендрограмма}
  \begin{center}
    \includegraphics[height=\textheight, width=\textwidth, keepaspectratio = true]{images/dendrogram}    
  \end{center}
\end{frame}

\begin{frame}{Дендрограмма}
  \begin{center}
    \includegraphics[height=0.8 \textheight, width=0.8 \textwidth, keepaspectratio = true]{images/dendrogram1}    
  \end{center}
\end{frame}

\begin{frame}{Вопрос}
  \centering  
  Может ли так случиться, что дендрограмма имеет самопересечения?
\end{frame}

\begin{frame}{Свойство монотонности}
  Кластеризация монотонна, если на каждом шаге расстояние $\rho$ между объединяемыми кластерами не уменьшается.\\
  \bigbreak  
  $$\rho_2 \leq \rho_3 \leq \dots \leq \rho_l$$
\end{frame}

\section{Обучение с частичным привлечением учителя}

\begin{frame}[standout]
  Вопросы?
\end{frame}

\appendix

\begin{frame}\frametitle{Что почитать по этой лекции}
  \begin{itemize}
    \item Christopher M. Bishop "Pattern Recognition and Machine Learning" Chapter 9
    \item G. James, D. Witten, T. Hastie, R. Tibshirani "An Introduction to Statistical Learning" Chapter 10.3
  \end{itemize}
\end{frame}

\begin{frame}\frametitle{Что происходит сейчас в области}
  ICML'16: \href{https://arxiv.org/pdf/1602.03258.pdf}{Interactive Bayesian Hierarchical Clustering}\\
  \bigbreak
  ICML'16: \href{https://arxiv.org/pdf/1602.01198.pdf}{k-variates++: more pluses in the k-means++}\\
  \bigbreak
  NIPS'16: \href{https://arxiv.org/pdf/1606.02404}{Clustering with Same-Cluster Queries}\\
  \bigbreak
  NIPS'16: \href{https://las.inf.ethz.ch/files/bachem16fast.pdf}{Fast and Provably Good Seedings for k-Means}\\
\end{frame}

\begin{frame}{На следующей лекции}
  	\begin{enumerate} [--]
		\item Деревья принятия решений
		\item Виды правил
		\item Поиск информативных закономерностей		
		\item Подрезание деревьев	
		\item Oblivious деревья
		\item Random forest
	\end{enumerate}
\end{frame}

\end{document}