\documentclass[10pt]{beamer}

\usepackage[T2A]{fontenc}
\usepackage[utf8]{inputenc}
\usepackage[russian,english]{babel}
\usepackage{subfig}
\usefonttheme[onlymath]{serif}

\usetheme[progressbar=frametitle]{metropolis}
\usepackage{appendixnumberbeamer}

\usepackage{booktabs}
\usepackage[scale=2]{ccicons}

\usepackage{pgfplots}
\usepgfplotslibrary{dateplot}

\usepackage{xspace}
\newcommand{\themename}{\textbf{\textsc{metropolis}}\xspace}
\newcommand{\TODO}[1]{\textbf{\textcolor{red}{TODO: #1}}}

\date{}
\author{Екатерина Тузова}


\title{Лекция 3}
\subtitle{Кластеризация}

\begin{document}

\maketitle

\section{Разбор летучки}

\section{Мотивирующий пример}

{\setbeamertemplate{frame footer}{\href{https://www.kaggle.com/abcsds/pokemon}{Pokemon with stats (https://www.kaggle.com/abcsds/pokemon)}}
\begin{frame}{Мотивирующий пример}
	\begin{figure}
	    \centering
	    \subfloat{{\includegraphics[width=2cm]{../lecture2/images/Bulbasaur} }}
	    \qquad
	    \subfloat{{\includegraphics[width=2cm]{../lecture2/images/Mewtwo} }}
    	    \qquad
    	    \subfloat{{\includegraphics[width=2cm]{../lecture2/images/Volcanion} }}
    	    \qquad
    	    \subfloat{{\includegraphics[width=2cm]{../lecture2/images/Ekans} }}
    	    \qquad
    	    \subfloat{{\includegraphics[width=2cm]{../lecture2/images/Nidorina} }}
	    \qquad 
    	    \subfloat{{\includegraphics[width=2cm]{../lecture2/images/Rattata} }}
	    \qquad
    	    \subfloat{{\includegraphics[width=2cm]{../lecture2/images/Sandshrew} }}
	    \qquad
    	    \subfloat{{\includegraphics[width=2cm]{../lecture2/images/Articuno} }}    	        	    
	\end{figure}
\end{frame}
}

\begin{frame}{Датасет}
    \centering
	\includegraphics[width=\textwidth]{../lecture2/images/pokemons}
\end{frame}

\begin{frame}{Постановка задачи кластеризации}
  Кластеризация -- задача разделения объектов одной природы на несколько групп так, чтобы объекты в одной группе обладали одним и тем же свойством.\\
  \bigbreak
  Кластеризация -- это обучение без учителя.
\end{frame}

\begin{frame}{Постановка задачи кластеризации}
	$X$ -- пространство объектов\\
	$\rho: X \times X \rightarrow [0, \infty)$ -- функция расстояния между объектами\\
	\bigbreak
	\alert{Найти}:\\
	$Y$ -- множество кластеров \\
	$a: X \rightarrow Y$ -- алгоритм кластеризации
\end{frame}


\section{Степени свободы в постановке задачи}

\begin{frame}{Степени свободы в постановке задачи}
	\begin{itemize} [<+- | alert@+>]
		\item[--] Критерий качества кластеризации
		\item[--] Число кластеров неизвестно заранее
		\item[--] Результат кластеризации существенно зависит от метрики
	\end{itemize}
\end{frame}

\section{Цели кластеризации}

\begin{frame}{Цели кластеризации}
	\begin{itemize} [<+- | alert@+>]
		\item[--] Сократить объём хранимых данных
		\item[--] Выделить нетипичные объекты
		\item[--] Упростить дальнейшую обработку данных
		\item[--] Построить иерархию множества объектов				
	\end{itemize}
\end{frame}

\section{Какие бывают кластеры?}

\begin{frame}{Сгущения}
	\begin{center}
    \includegraphics[height=0.6 \textheight, keepaspectratio = true]{images/cluster1}  
	\end{center}
\end{frame}


\begin{frame}{Ленты}
	\begin{center}
	  \includegraphics[height=0.6 \textheight, keepaspectratio = true]{images/cluster2}  
	\end{center}
\end{frame}

\begin{frame}{С центром}
	\begin{center}
	  \includegraphics[height=0.6 \textheight, keepaspectratio = true]{images/cluster3}  
	\end{center}
\end{frame}

\begin{frame}{С перемычками}
	\begin{center}
	  \includegraphics[height=0.6 \textheight, keepaspectratio = true]{images/cluster4}  
	\end{center}
\end{frame}

\begin{frame}{На фоне}
	\begin{center}
	  \includegraphics[height=0.6 \textheight, keepaspectratio = true]{images/cluster5}  
	\end{center}
\end{frame}

\begin{frame}{Перекрывающиеся}
	\begin{center}
	  \includegraphics[height=0.6 \textheight, keepaspectratio = true]{images/cluster6}  
	\end{center}
\end{frame}

\begin{frame}{Чувствительность к выбору метрики}
	\begin{center}
	  \includegraphics[height=0.8 \textheight, keepaspectratio = true]{images/weight_height1}  
	\end{center}
\end{frame}

\begin{frame}{Чувствительность к выбору метрики}
	\begin{center}
	  \includegraphics[height=0.8 \textheight, keepaspectratio = true]{images/weight_height2}  
	\end{center}
\end{frame}

\begin{frame}{Чувствительность к выбору метрики}
	\begin{center}
	  \includegraphics[height=0.8 \textheight, keepaspectratio = true]{images/weight_height3}  
	\end{center}
\end{frame}

\begin{frame}{Чувствительность к выбору метрики}
	\begin{center}
	  \includegraphics[height=0.8 \textheight, keepaspectratio = true]{images/weight_height4}  
	\end{center}
\end{frame}

\section{Оценка качества кластеризации}

\begin{frame}{Оценка качества кластеризации}
  \alert{Идея}: Минимизировать среднее внутрикластерное расстояние и при этом максимизировать среднее межкластерное расстояние.
\end{frame}

\begin{frame}{Оценка качества кластеризации}
  \alert{Идея}: Минимизировать среднее внутрикластерное расстояние и при этом максимизировать среднее межкластерное расстояние.
	\bigbreak
	$${\frac{\sum\limits_{a(x_i) = a(x_j)} \rho(x_i, x_j)}{\sum\limits_{a(x_i) = a(x_j)} 1} \rightarrow \min}$$
	\bigbreak
	$${\frac{\sum\limits_{a(x_i) \neq a(x_j)} \rho(x_i, x_j)}{\sum\limits_{a(x_i) \neq a(x_j)} 1} \rightarrow \max}$$
\end{frame}


\begin{frame}{Методы кластеризации}
	\begin{enumerate} [-]
		\item Иерархические
		\item Графовые 
		\item Статистические 
	\end{enumerate}
\end{frame}

\section{Иерархическая кластеризация}

\begin{frame}{Агломеративный алгоритм Ланса-Уильямса}
	\alert{Идея}:\\
	\begin{enumerate}
		\item Считаем каждую точку кластером. 
		\item Затем объединяем ближайшие точки в новый кластер. 
		\item Повторяем.
	\end{enumerate}
\end{frame}


\begin{frame}{Алгоритм Ланса-Уильямса}
	${C_1 = \left\{ \left\{ x_1\right\}, \left\{x_2 \right\}, \dots, \left\{x_l \right\} \right\}}$\\
	for ${t=2, \dots, l }$:\\
	\hspace{5mm} ${(U, V) = \arg\min\limits_{U \neq V} \rho(U, V)}$\\
	\hspace{5mm} $W = U \cup V$\\
	\hspace{5mm} ${C_t = C_{t-1} \cup \left\{ W \right\}\setminus \left\{U, V \right\} }$\\
	\hspace{5mm} foreach ${S \in C_t}$\\
	\hspace{10mm}   compute $\rho(W, S)$\\
\end{frame}


\begin{frame}{Алгоритм Ланса-Уильямса}
  \begin{center}
    Чего-то не хватает?  
  \end{center}
\end{frame}

\begin{frame}\frametitle{Формула Ланса-Уильямса}
	\begin{minipage}[t]{0.45\linewidth}
    \vspace{15mm}
 		${ W = \left\{ U \cup V \right\} }$\\	
		\bigbreak		
		\alert{Знаем}:\\
		${\rho(U, S), \rho(V, S), \rho(U, V)}$
		\bigbreak				
		Расстояние $\rho(W, S)$?\\	
	\end{minipage}%
	\begin{minipage}[t]{0.55\linewidth}
	  \begin{figure}[htbp]
	    \includegraphics[height=150pt, keepaspectratio = true]{images/lans-formula}  
	  \end{figure}
  \end{minipage}%
\end{frame}

\begin{frame}{Формула Ланса-Уильямса}
	${ W = \left\{ U \cup V \right\} }$\\
	\bigbreak
	${\rho(U \cup V, S) = \alpha_U \rho(U, S) + \alpha_V \rho(V, S) + }$ \\
	\hspace{30mm} ${ + \beta \rho(U, V) + \gamma \vert \rho(U, S) - \rho(V, S)\vert}$\\
	\bigbreak
	${\alpha_U, \alpha_V, \beta, \gamma}$ -- числовые параметры
\end{frame}

\begin{frame}{Параметры}
	Значения параметров
	${\alpha_U, \alpha_V, \beta, \gamma}$ ?
\end{frame}

\begin{frame}{Расстояние ближнего соседа}
	\begin{figure}[htbp]
	  \includegraphics[height=150pt, keepaspectratio = true]{images/lans1}  
	\end{figure}
\end{frame}

\begin{frame}{Расстояние ближнего соседа}
	Расстояние ближнего соседа:\\
	\begin{figure}[htbp]
	  \includegraphics[height=150pt, keepaspectratio = true]{images/lans1}  
	\end{figure}
	${\alpha_U = \alpha_V = \frac{1}{2}}$ \\${\beta = 0}$ \\${\gamma = -\frac{1}{2}}$
\end{frame}

\begin{frame}{Расстояние дальнего соседа}
	Расстояние дальнего соседа:\\
	\begin{figure}[htbp]
	  \includegraphics[height=150pt, keepaspectratio = true]{images/lans2}  
	\end{figure}
\end{frame}

\begin{frame}{Расстояние дальнего соседа}
	\begin{figure}[htbp]
	  \includegraphics[height=150pt, keepaspectratio = true]{images/lans2}  
	\end{figure}
	${\alpha_U = \alpha_V = \frac{1}{2}}$ \\${\beta = 0}$ \\${\gamma = \frac{1}{2}}$
\end{frame}

\begin{frame}{Групповое среднее}
	\begin{figure}[htbp]
	  \includegraphics[height=150pt, keepaspectratio = true]{images/lans3}  
	\end{figure}
\end{frame}

\begin{frame}{Групповое среднее}
	\begin{figure}[htbp]
	  \includegraphics[height=150pt, keepaspectratio = true]{images/lans3}  
	\end{figure}
	${\alpha_U = \frac{\vert U \vert}{\vert W \vert}}$\\${\alpha_V = \frac{\vert V \vert}{\vert W \vert}}$ \\${\beta = \gamma = 0}$
\end{frame}

\begin{frame}{Расстояние Уорда}
	\begin{figure}[htbp]
	  \includegraphics[height=150pt, keepaspectratio = true]{images/lans4}  
	\end{figure}

  ${\alpha_U = \frac{\vert S \vert + \vert U \vert}{\vert S \vert + \vert W \vert}}$\\${\alpha_V = \frac{\vert S \vert + \vert V \vert}{\vert S \vert + \vert W \vert}}$ \\${\beta = \frac{ -\vert S \vert}{\vert S \vert + \vert W \vert}}$ \\${\gamma = 0}$
\end{frame}

\section{Визуализация кластеров}

\begin{frame}{Диаграмма вложения}
  \begin{center}
    \includegraphics[height=0.8 \textheight, keepaspectratio = true]{images/diagram}    
  \end{center}
\end{frame}

\begin{frame}{Дендрограмма}
  \begin{center}
    \includegraphics[height=0.8 \textheight, width=0.8 \textwidth, keepaspectratio = true]{images/dendrogram}    
  \end{center}
\end{frame}

\begin{frame}{Дендрограмма}
  \begin{center}
    \includegraphics[height=0.8 \textheight, width=0.8 \textwidth, keepaspectratio = true]{images/dendrogram1}    
  \end{center}
\end{frame}

\begin{frame}{Вопрос}
  \centering  
  Может ли так случиться, что дендрограмма имеет самопересечения?
\end{frame}

\begin{frame}{Свойство монотонности}
  Кластеризация монотонна, если на каждом шаге расстояние $\rho$ между объединяемыми кластерами не уменьшается.\\
  \bigbreak  
  $$\rho_2 \leq \rho_3 \leq \dots \leq \rho_l$$
\end{frame}

\section{Графовые алгоритмы}

\begin{frame}{Графовые алгоритмы}
  \begin{center}
    Какие есть две очевидные идеи?
  \end{center}    
\end{frame}

\begin{frame}{Графовые алгоритмы}
	Идеи:\\
	\begin{enumerate}
		\item Выделение связных компонент
		\item Минимальное покрывающее дерево
	\end{enumerate}
\end{frame}

\begin{frame}{Выделение связных компонент}
	\begin{enumerate}
		\item Рисуем полный граф с весами, равными расстоянию между объектами
		\item Выбираем лимит расстояния $r$ и выкидываем все ребра длиннее $r$
		\item Компоненты связности полученного графа -- наши кластеры
	\end{enumerate}
\end{frame}

\begin{frame}{Выделение связных компонент}
  \begin{center}
    Как искать \alert{компоненты связности}?
  \end{center}
\end{frame}

\begin{frame}{Минимальное покрывающее дерево}
	Минимальное остовное дерево -- дерево, содержащее все вершины графа и имеющее минимальный суммарный вес ребер.\\
\end{frame}

\begin{frame}{Минимальное покрывающее дерево}
  Как использовать минимальное остовное дерево для разбиения на кластеры?
\end{frame}

\begin{frame}{Минимальное покрывающее дерево}
	Строим минимальное остовное дерево, а потом выкидываем из него ребра максимального веса.\\
	\bigbreak
	Сколько ребер выбросим -- столько кластеров получим.
\end{frame}

\section{Статистические алгоритмы}

\begin{frame}{Алгоритм FOREL}
  \alert{Идея}:\\
	\begin{itemize}
		\item[--] Выделить все точки выборки $x_i$, попадающие внутрь сферы $\rho(x_i, x_0) \leq R$
		\item[--] Перенести $x_0$ в центр тяжести выделенных точек
		\item[--] Повторять пока $x_0$ не стабилизируется
	\end{itemize}
\end{frame}

\begin{frame}{Алгоритм FOREL}
	Input: X, R\\
	${U = X, C = 0}$\\\vspace{2mm}
	while ${U \neq 0}$:\\
	\hspace{5mm} выбрать случайную точку $x_0$\\
	\vspace{2mm}
	\hspace{5mm} Повторять пока $x_0$ не стабилизируется:\\
	\vspace{2mm}
	\hspace{10mm} ${c = \left\{ x \in X \vert \rho(x, x_0) < R \right\}}$ \\
	\vspace{2mm}
	\hspace{10mm} $x_0 = \frac{1}{\vert c \vert} \sum\limits_{x_i \in c} x_i$\\
	\vspace{2mm}
	\hspace{5mm} ${U = U \setminus c}$, ${C = C \cup \left\{ c \right\}}$
\end{frame}

\begin{frame}{Алгоритм FOREL}
	\begin{itemize} [<+- | alert@+>]
		\item[+] Наглядность
		\item[+] Сходимость
		\bigbreak
		\item[--] Зависимость от выбора $x_0$
		\item[--] Плохо работает, если изначальная выборка плохо делится на кластеры
	\end{itemize}
\end{frame}

% можно какую-нибудь иллюстрацию придумать

\begin{frame}{Метод $k$-средних}
	\alert{Идея}:  минимизировать меру ошибки\\
	\bigbreak
	${E(X, C) = \sum\limits_{i = 1}^n \Vert x_i -\mu_i \Vert^2}$\\
	\bigbreak
	$\mu_i$ -- ближайший к $x_i$ центр кластера
\end{frame}

\begin{frame}{Метод $k$-средних}
	Инициализировать центры $k$ кластеров \\
	\vspace{2mm}
	Пока $c_i$ не перестанет меняться:\\
	\hspace{5mm} $c_i = \arg\min\limits_{c \in C} \rho(x_i, \mu_c)$ \hspace{5mm} $i = 1,\dots, l$\\
	\vspace{2mm}\hspace{5mm} ${\mu_c = \frac{\sum\limits_{c_i = c} f_j(x_i)}{\sum\limits_{c_i = c} 1} }$ \hspace{10mm} $j = 1,\dots, n$, $c \in C$\\
	\vspace{2mm}
	$\mu_c$ -- новое положение центров кластеров\\
	$c_i$ -- принадлежность $x_i$ к кластеру\\
	$\rho(x_i, \mu_c)$ -- расстояние от $x_i$ до центра кластера $\mu_c$
\end{frame}

\begin{frame}{Метод $k$-средних}
  \TODO{картинка}
%	\begin{figure}[htbp]
%	  \includegraphics[height=190pt, keepaspectratio = true]{images/k-means-1}   
%	\end{figure}
\end{frame}

\begin{frame}{Метод $k$-средних}
\TODO{картинка}
%	\begin{figure}[htbp]
%	  \includegraphics[height=190pt, keepaspectratio = true]{images/k-means-2}   
%	\end{figure}
\end{frame}

\begin{frame}{Метод $k$-средних}
\TODO{картинка}
%	\begin{figure}[htbp]
%	  \includegraphics[height=190pt, keepaspectratio = true]{images/k-means-3}   
%	\end{figure}
\end{frame}

\begin{frame}{Метод $k$-средних}
\TODO{картинка}
%	\begin{figure}[htbp]
%	  \includegraphics[height=190pt, keepaspectratio = true]{images/k-means-4}   
%	\end{figure}
\end{frame}

\begin{frame}{Метод $k$-средних}
\TODO{картинка}
%	\begin{figure}[htbp]
%	  \includegraphics[height=190pt, keepaspectratio = true]{images/k-means-5}   
%	\end{figure}
\end{frame}

\begin{frame}{Особенности метода $k$-средних}
	\begin{itemize}
		\item[--] Чувствительность к начальному выбору $\mu_c$
		\item[--] Необходимость задавать $k$
	\end{itemize}
\end{frame}

\begin{frame}{Чувствительность к начальному выбору $\mu_c$}
\TODO{картинка}
%	\begin{figure}[htbp]
%	  \includegraphics[height=180pt, keepaspectratio = true]{images/local_min2}  
%	\end{figure}
\end{frame}

\begin{frame}{Чувствительность к начальному выбору $\mu_c$}
\TODO{картинка}
%	\begin{figure}[htbp]
%	  \includegraphics[height=180pt, keepaspectratio = true]{images/local_min4}  
%	\end{figure}
\end{frame}

\begin{frame}{Чувствительность к начальному выбору $\mu_c$}
\TODO{картинка}
%	\begin{figure}[htbp]
%	  \includegraphics[height=180pt, keepaspectratio = true]{images/local_min6}  
%	\end{figure}
\end{frame}

\begin{frame}{Чувствительность к начальному выбору $\mu_c$}
\TODO{картинка}
%	\begin{figure}[htbp]
%	  \includegraphics[height=180pt, keepaspectratio = true]{images/local_min7}  
%	\end{figure}
\end{frame}

\begin{frame}{Необходимость задавать $k$}
\TODO{картинка}
%	\begin{figure}[htbp]
%	  \includegraphics[height=180pt, keepaspectratio = true]{images/k_means_k}  
%	\end{figure}
\end{frame}

\section{Устранение недостатков}

\begin{frame}{Устранение недостатков}
	\begin{itemize} [<+- | alert@+>]
		\item[--] Несколько случайных кластеризаций
		\item[--] Постепенное наращивание числа $k$
		\item[--] Использование k-means++
	\end{itemize}
\end{frame}

\begin{frame}{k-means++}
	\begin{enumerate}
		\item Выбрать первый центроид случайным образом
		\item Для каждой точки найти значение квадрата расстояния до ближайшего центроида.
		\item Выбрать из этих точек следующий центроид так, чтобы вероятность выбора точки была пропорциональна вычисленному для неё квадрату расстояния
	\end{enumerate}
\end{frame}

\begin{frame}{X-means}
	\alert{Идея}:\\
	\begin{enumerate}
		\item Получать на вход не k, а диапазон, в котором может находиться k.
		\item Запустить k-means на самом маленьком значении из диапазона.
		\item Разбить пополам полученные кластеры и проверить, не улучшилась ли кластеризация.
	\end{enumerate}
\end{frame}

\begin{frame}{X-means}
\TODO{картинка}
%	\begin{figure}[htbp]
%	  \includegraphics[height=140pt, keepaspectratio = true]{images/x-means}  
%	    \includegraphics[height=140pt, keepaspectratio = true]{images/x-means-1}
%	\end{figure}
https://www.cs.cmu.edu/~dpelleg/download/xmeans.pdf
\end{frame}

\begin{frame}{X-means}
  \begin{center}
    Как проверить, что кластеризация улучшилась?
  \end{center}
\end{frame}

% понять, как про это говорить
\begin{frame}{Байесовский информационный критерий}
	$BIC_j = L_j(X)  + \frac{d}{2} \log(l)$\\
	\bigbreak
	$L_j$ -- логарифмическая функция правдоподобия для $j$-й модели \\
	$d$ -- длина вектора параметров\\
	$l$ -- количество объектов в выборке\\
\end{frame}

\section{Недостатки k-means}

\begin{frame}{"Не сферические данные"}
  \TODO{картинка}
%	\begin{figure}[htbp]
%	  \includegraphics[height=180pt, keepaspectratio = true]{images/non_spherical-1}  
%	\end{figure}
\end{frame}	

\begin{frame}{"Не сферические данные"}
	\TODO{картинка}
%	\begin{figure}[htbp]
%	  \includegraphics[height=180pt, keepaspectratio = true]{images/non_spherical-2}  
%	\end{figure}
\end{frame}

\begin{frame}{Разноразмерные кластеры}
\TODO{картинка}
%	\begin{figure}[htbp]
%	  \includegraphics[height=180pt, keepaspectratio = true]{images/different_sizes-1}  
%	\end{figure}
\end{frame}

\begin{frame}{Разноразмерные кластеры}
\TODO{картинка}
%	\begin{figure}[htbp]
%	  \includegraphics[height=180pt, keepaspectratio = true]{images/different_sizes-2}  
%	\end{figure}
\end{frame}

\begin{frame}[standout]
  Вопросы?
\end{frame}

\appendix

\begin{frame}{На следующей лекции}
  \TODO{what next}
\end{frame}

\end{document}