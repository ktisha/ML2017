\documentclass[10pt]{beamer}

\usepackage[T2A]{fontenc}
\usepackage[utf8]{inputenc}
\usepackage[russian,english]{babel}
\usepackage{subfig}
\usefonttheme[onlymath]{serif}

\usetheme[progressbar=frametitle]{metropolis}
\usepackage{appendixnumberbeamer}

\usepackage{booktabs}
\usepackage[scale=2]{ccicons}

\usepackage{pgfplots}
\usepgfplotslibrary{dateplot}

\usepackage{xspace}
\newcommand{\themename}{\textbf{\textsc{metropolis}}\xspace}
\newcommand{\TODO}[1]{\textbf{\textcolor{red}{TODO: #1}}}

\date{}
\author{Екатерина Тузова}

\usepackage{tikz}
\usetikzlibrary{arrows,positioning} 
\tikzset{
    mynode/.style={rectangle,rounded corners,draw=black, top color=white,very thick, inner sep=1em, text centered},
    mycircle/.style={circle,draw=black, top color=white,very thick, text centered},
    pil/.style={ ->, thick, shorten <=2pt, shorten >=2pt,}
}  

\title{Лекция 7}
\subtitle{Выбор моделей}

\begin{document}

\maketitle

\section{Разбор летучки}

\begin{frame}{Задача выбора метода обучения}
  $X$ - множество объектов \\
	$Y$ - множество классов \\
	Обучающая выборка: ${X^l = (x_i, y_i)_{i=1}^l}$ \\ 
	Целевая функция: $f: X \rightarrow Y$\\
	\bigbreak
	Набор моделей алгоритмов $A_t: X \rightarrow Y$, $t \in T$ \\
	Методы обучения $\mu_t: X \times Y \rightarrow A_t$, $t \in T$ \\
	\bigbreak
	\alert{Задача}: Найти алгоритм $a \in A_t$  с наилучшей обобщающей способностью
\end{frame}

\begin{frame}{Модель}
  \begin{tikzpicture}[node distance=1cm, scale=0.8, transform shape]
    \node[mynode, text width=5cm] (target) 
      {Неизвестная целевая функция\\
      \textcolor{blue}{$f: X \rightarrow Y$}};
    \node[mynode, text width=3.5cm, below=1cm of target] (training) 
      {Обучающая выборка\\
      \textcolor{blue}{${X^l = (x_i, y_i)_{i=1}^l}$}}
          edge[pil, <-] (target.south);
    \node[below=1cm of training] (dummy) {}; 
    \onslide<2, 3>{    
      \node[mycircle,text width=1.5cm, below=1cm, right=2cm of dummy] (learning) 
        {Метод обучения\\
        \textcolor{blue}{$\mu_t$}}
        edge[pil, <-, bend left=15] (training.south);  
      }
    \onslide<3>{   
      \node[mynode, text width=3cm, below=1cm of dummy] (hypothesis) 
        {Набор моделей\\
        \textcolor{blue}{$A_t$}}
        edge[pil, bend left=15] (learning.west);  
        }
    \node[mynode, text width=3.5cm, right=1cm of learning] (final) 
      {Финальный алгоритм\\
      \textcolor{blue}{$a \in A_t$}}
      edge[pil, <-] (learning.east);    
  \end{tikzpicture}
\end{frame}

\begin{frame}{Модель}
  \begin{tikzpicture}[node distance=1cm, scale=0.8, transform shape]
    \node[mynode, text width=5cm] (target) 
      {Неизвестная целевая функция\\
      \textcolor{blue}{$f: X \rightarrow Y$}};
    \node[mynode, text width=3.5cm, below=1cm of target] (training) 
      {Обучающая выборка\\
      \textcolor{blue}{${X^l = (x_i, y_i)_{i=1}^l}$}}
          edge[pil, <-] (target.south);
    \node[below=1cm of training] (dummy) {}; 
    \node[mycircle, orange, text width=1.5cm, below=1cm, right=2cm of dummy] (learning) 
      {Метод обучения\\
      \textcolor{blue}{$\mu_t$}}
      edge[pil, <-, bend left=15] (training.south);  
    \node[mynode, orange, text width=3cm, below=1cm of dummy] (hypothesis) 
      {Набор моделей\\
      \textcolor{blue}{$A_t$}}
      edge[pil, bend left=15] (learning.west);  
    \node[mynode, text width=3.5cm, right=1cm of learning] (final) 
      {Финальный алгоритм\\
      \textcolor{blue}{$a \in A_t$}}
      edge[pil, <-] (learning.east);    
  \end{tikzpicture}
\end{frame}

\section{Пример}

\begin{frame}{Перцептрон}
  Набор моделей:\\
  $a(\mathbf{x}) = \sign(\sum\limits_{j=1}^n {\only<2,3>{\color{orange}}w_j} x^j - {\only<2,3>{\color{orange}}w_0})$\\
  \bigbreak
  \pause
  \pause
  Метод обучения:\\
  \begin{algorithmic}[1]
    \Function{perceptron}{$X^l$}
      \State Инициализировать ${w_0, \dots, w_n}$
      \MRepeat [пока $\mathbf{w}$ изменяются] 
         \For {$i = 1, \dots, l$}
           \If {$a(x_i) \neq y_i$}
             \State $\mathbf{w} = \mathbf{w} + y_i \mathbf{x_i}$
           \EndIf  
         \EndFor
       \EndRepeat
     \EndFunction
    \end{algorithmic}
\end{frame}

\begin{frame}{Задача}  
  \centering
  Научиться оценивать метод обучения и обобщающую способность алгоритма
\end{frame}

{\foot{loss function}
\begin{frame}{Функция потерь}
  Функция потерь $\mathcal{L}(a, x_i) $ -- характеризует величину ошибки алгоритма $a$ на объекте $x_i$.\\
  \bigbreak
  Если $\mathcal{L} (a, x_i) = 0$, то ответ $a(x_i)$ называется корректным.
\end{frame}
}

{\foot{loss function}
\begin{frame}{Функция потерь}
  Задача классификации:\\
  $\mathcal{L}(a, x_i) = [a(x_i) \neq y_i]$ -- индикатор ошибки\\
  \bigbreak
  Задача регрессии:\\
  $\mathcal{L}(a, x_i) = \vert a(x_i) - y_i \vert$ -- абсолютное значение ошибки\\
  $\mathcal{L}(a, x_i) = (a(x_i) - y_i)^2$ -- квадратичная ошибка\\
\end{frame}
}

{\foot{функционал средних потерь, Empirical Risk}
\begin{frame}{Функционал качества}
  Функционал качества алгоритма $a$ на выборке $X^l$:\\
  $$Q(a, X^l) = \frac{1}{l} \sum\limits_{i=1}^l \mathcal{L}(a, x_i)$$\\
  Минимизация эмпирического риска:\\
  $$\arg\min\limits_{A_t} Q(a, X^l)$$
\end{frame}
}

\begin{frame}{Внутренний функционал}  
  $$Q_{\mu}(X^l) = Q(\mu(X^l), X^l)$$\\
  \bigbreak
  Этот функционал оценивает качество обучения на выборке $X^l$\\
  \bigbreak
  \pause
  Какая с этим может быть проблема?
\end{frame}

\begin{frame}{Простая аналогия}
  \centering
  С какой вероятностью монета, подброшенная 10 раз, выпадет одной и той же стороной все 10 раз?
  \pause
  \bigbreak
  \textcolor{blue}{$0.001$}
\end{frame}

\begin{frame}{Простая аналогия}
  \centering
  С какой вероятностью одна из 1000 монет, каждая из которых подброшена 10 раз, выпадет одной и той же стороной все 10 раз?
  \pause
  \bigbreak
  \textcolor{blue}{$0.63$}
\end{frame}

\begin{frame}{Внешний функционал}  
  $$Q_{\mu}(X^l) = Q(\mu(X^l), X^l)$$\\
  \bigbreak
  Внешний функционал по отложенной выборке:\\
  $$Q_{\mu}(X^t, X^k) = Q(\mu(X^t), X^k)$$\\
  \bigbreak
  \onslide<2>{
    Какой здесь недостаток?\\
  }
  \onslide<3>{
    Сильная зависимость от разбиения $X^l = X^t \sqcup X^k$
  }
\end{frame}

{\foot{Полный скользящий контроль, cross-validation}
\begin{frame}{Кроссвалидация}  
  \alert{Идея}: Усреднить по всем $C_l^t$ выборкам $X^l = X^t \sqcup X^k$\\
  $$CCV(\mu, X^l) = \frac{1}{C_l^t} \sum\limits_{X^t} Q_{\mu}(X^t, X^k)$$\\
  \bigbreak
  Во что превратится оценка при $k=1$?
\end{frame}
}

\begin{frame}{Недостатки}  
  \begin{enumerate}
    \item[--] Оценка вычислительно слишком сложна
    \item[--] Не учитывает дисперсию $X^k$
  \end{enumerate}
\end{frame}

\begin{frame}{k-fold кроссвалидация}  
  \alert{Идея}: Возьмём случайное разбиение $X^l = X_1 \sqcup \dots \sqcup X_k$ на $k$ блоков равной длины.\\
  $$CV_k(\mu, X^l) = \frac{1}{k} \sum\limits_{i=1}^{k} Q_{\mu}(X^l \setminus X_i, X_i)$$\\
  \bigbreak
  \pause
  Недостатки:  
  \begin{enumerate}
    \item[--] Оценка зависит от разбиения на блоки
    \item[--] Каждый объект только один раз участвует в контроле    
  \end{enumerate}
\end{frame}

{\foot{Repeated k-fold cross validation}
\begin{frame}{Повторяющийся k-fold CV}  
  \alert{Идея}: Выборка разбивается $t$ раз случайным образом на $k$ блоков\\
  
  $$CV_{tk}(\mu, X^l) = \frac{1}{t} \sum\limits_{j=1}^{t} \frac{1}{k} \sum\limits_{i=1}^{k} Q_{\mu}(X^l \setminus X_{ji}, X_{ji})$$\\
  \bigbreak
  \pause
  \begin{enumerate}
    \item[+] Выбором $t$ можно улучшать точность оценки
    \item[+] Каждый объект участвует в контроле $t$ раз
  \end{enumerate}
\end{frame}
}

\begin{frame}{Критерий непротиворечивости моделей}  
  \alert{Идея}: Если модель верна, то алгоритмы, настроенные по разным частям данных, не должны противоречить друг другу.\\
\end{frame}

{\foot{Learning curve}
\begin{frame}{Кривая обучения}  
  \TODO{картинка}
\end{frame}
}

\section{Аналитический подход}

\begin{frame}{Как использовать аналитические оценки}  
  \begin{enumerate}
    \item Получить верхнюю оценку вероятности переобучения $R_{\varepsilon}$\\
      $$R_{\varepsilon}(\mu, X^l) = P[Q_{\mu}(X^t, X^k) - Q_{\mu}(X^t) \geq \varepsilon] \leq \eta(\varepsilon, A)$$
    \item Тогда для любых $X^l$, $A$, $\mu$ и $\eta$ с вероятностью не менее $1-\eta$ справедливо\\
      $$Q_{\mu}(X^t, X^k) \leq Q_{\mu}(X^t) + \varepsilon(\eta, A)$$
    \item Будем оптимизивать \\
      $$Q_{\mu}(X^l) + \varepsilon(\eta, A) \rightarrow \min\limits_{\mu}$$
  \end{enumerate}
\end{frame}

\begin{frame}{Регуляризация}  
  Регуляризатор -- добавка к внутреннему критерию, обычно штраф за сложность модели.\\
  \TODO{примеры для линейной модели, ридж, лассо ...}
\end{frame}

\begin{frame}{Разновидности $L_0$ регуляризации}  
  \TODO{акаике, байес, vc}
\end{frame}

\begin{frame}{Выбор модели по нескольким критериям}  
  \TODO{картинка}
\end{frame}

\begin{frame}{Генерация модельных данных}  

\end{frame}

\begin{frame}[standout]
  Вопросы?
\end{frame}

\appendix

\begin{frame}\frametitle{На следующей лекции}
	\begin{itemize}
    	\item[--] Функционалы качества
    	\item[--] Неравенство Хефдинга
    	\item[--] Близость гипотез
    	\item[--] Неравенство Вапника-Червоненкиса
    	\item[--] Генерация модельных данных    	    	
	\end{itemize}
\end{frame}
\end{document}

\end{document}