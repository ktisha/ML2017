\documentclass[10pt]{beamer}

\usepackage[T2A]{fontenc}
\usepackage[utf8]{inputenc}
\usepackage[russian,english]{babel}
\usepackage{subfig}
\usefonttheme[onlymath]{serif}

\usetheme[progressbar=frametitle]{metropolis}
\usepackage{appendixnumberbeamer}

\usepackage{booktabs}
\usepackage[scale=2]{ccicons}

\usepackage{pgfplots}
\usepgfplotslibrary{dateplot}

\usepackage{xspace}
\newcommand{\themename}{\textbf{\textsc{metropolis}}\xspace}
\newcommand{\TODO}[1]{\textbf{\textcolor{red}{TODO: #1}}}

\date{}
\author{Екатерина Тузова}


\title{Лекция 13}
\subtitle{Композиции алгоритмов}

\begin{document}

\section{Разбор летучки}

\maketitle

\section{Мотивация}

\section{Где уже видели композиции?}

\begin{frame}{Определение композиции}
  ${X^l = (x_i, y_i)_{i = 1}^l}$ -- обучающая выборка\\
  $b:X \rightarrow R$ --- базовый алгоритм\\
  $C:R \rightarrow Y$ --- решающее правило\\
  $R$ --- пространство оценок.
  \bigbreak
  $a(x) = C(b(x))$\\
\end{frame}

\begin{frame}{Определение композиции}
  Композиция базовых алгоритмов $b_1, \dots, b_T$\\
  \bigbreak
  $a(x) = C(F(b_1(x), \dots, b_T(x)))$\\
  $F: R^T \rightarrow R$ -- корректирующая операция
\end{frame}

\begin{frame}{Примеры}
  \begin{enumerate}
    \item Простое голосование\\
      $$F(b_1(x), \dots, b_T(x)) = \frac{1}{T} \sum\limits_{t=1}^{T} b_t(x)$$
    \item Взвешенное голосование\\
      $$F(b_1(x), \dots, b_T(x)) = \sum\limits_{t=1}^{T} \alpha_t b_t(x)$$
    \item Смесь алгоритмов\\
      $$F(b_1(x), \dots, b_T(x)) = \sum\limits_{t=1}^{T} g_t(x) b_t(x)$$      
  \end{enumerate}
\end{frame}

{\foot{Boosting}
\begin{frame}{Бустинг}
  $Y = \{\pm 1\}$, \qquad $b_t: X\rightarrow \{-1, 0, +1\}$, \qquad $C(b) = \sign(b)$\\
  $b_t(x) = 0$ -- отказ от классификации\\
  \bigbreak
  \pause
  $a(x) = \sign(\sum\limits_{t=1}^{T} \alpha_t b_t(x))$\\
  \bigbreak
  Функционал качества композиции:\\
  $Q_T = \sum\limits_{i=1}^l [y_i \sum\limits_{t=1}^{T} \alpha_t b_t(x) < 0 ]$
\end{frame}
}

\begin{frame}{Adaboost}
  $B$ -- семейство базовых алгоритмов.\\
  \bigbreak
  Пусть для любого нормированного вектора весов $U^l$ существует алгоритм $b \in B$, классифицирующий выборку немного лучше, чем наугад $P(b, U^l) > N (b, U^l)$. 
  Минимум функционала $Q_T$ достигается при:\\
  $$b_T = \arg\max\limits_{b \in B} \sqrt{P(b, W^l)} - \sqrt{N(b, W^l)}$$
  $$a_T = \frac{1}{2} \ln \frac{P(b_T, W^l)}{N(b_T, W^l)}$$
\end{frame}

\begin{frame}{Рекоммендации}
  \begin{itemize}
    \item Чаще всего в качестве базовых классификаторов используются решающие деревья
    \item Для SVM бустинг не эффективен     
  \end{itemize}
\end{frame}

\begin{frame}{Обоснование бустинга}
    
\end{frame}

\begin{frame}{Наблюдения}
  \begin{itemize}[<+->]
    \item[+] Хорошая обобщающая способность
    \item[+] Простота реализации
    \item[+] Накладные расходы на построение не велики
    \bigbreak
    \item[--] Склонен к переобучению при наличии большого количества шума в данных
    \item[--] Требует большой обучающей выборки
    \item[--] Жадность приводит к неоптимальности
    
  \end{itemize}
\end{frame}

\begin{frame}{Бэггинг}
    
\end{frame}

\begin{frame}{Bagboo}
    
\end{frame}

\begin{frame}{Stacking}
    
\end{frame}

\begin{frame}{Voting}
    
\end{frame}

\begin{frame}{Почему эти подходы работают}
  \begin{enumerate}
    \item Бэггинг уменьшает разброс
    \item Бустинг уменьшает разброс и смещение
    \item Чем сильнее коррелируют базовые алгоритмы, тем менее эффективны композиции
  \end{enumerate}
\end{frame}

\begin{frame}[standout]
  Вопросы?
\end{frame}

\appendix

\begin{frame}\frametitle{На следующей лекции}
	\begin{itemize}
    	\item[--] Функционалы качества
    	\item[--] Неравенство Хефдинга
    	\item[--] Близость гипотез
    	\item[--] Неравенство Вапника-Червоненкиса
    	\item[--] Генерация модельных данных    	    	
	\end{itemize}
\end{frame}
\end{document}

\end{document}