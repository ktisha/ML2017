%% -*- TeX-engine: luatex; ispell-language: "russian" -*-

\documentclass[a4paper,12pt]{article}
\usepackage{subcaption}
\usepackage[left=1.5cm,right=2cm,top=1.5cm,bottom=2cm]{geometry}

\usepackage{parskip}
\setlength{\parindent}{0mm}
\setcounter{secnumdepth}{1}

\usepackage{amsmath}

\usepackage{fontspec}
\setmainfont{PT Serif}
\newfontfamily\cyrillicfont[Script=Cyrillic,Ligatures=TeX]{PT Serif}
\setsansfont{PT Sans}
\setmonofont[Ligatures=NoCommon]{PT Mono}
\defaultfontfeatures{Ligatures=TeX}

\usepackage[bold-style=ISO]{unicode-math}
\setmathfont{XITS Math}

\usepackage{microtype}

\usepackage{hyperref}

\usepackage{polyglossia}
\setmainlanguage{russian}
\setotherlanguage{english}

\usepackage{csquotes}

%% for code snippets
\usepackage{minted}
\newminted[pycon]{pycon}{fontsize=\footnotesize}
\newminted[python3]{python3}{fontsize=\footnotesize}
\newminted[bash]{bash}{fontsize=\footnotesize}
\newmintinline[pythoninline]{python3}{fontsize=\footnotesize}
\newmintinline[bashinline]{bash}{fontsize=\footnotesize}

\pagestyle{empty}


\begin{document}
\subsection*{Домашнее задание №3: <<Одеревенеть от страха>>}

\begin{tabular}{@{}lr}
  \textbf{Дедлайн 1} (20 баллов): & 23 марта, 23:59 \\
  \textbf{Дедлайн 2} (10 баллов): & 30 марта, 23:59
\end{tabular}

Домашнее задание нужно написать на Python и сдать в виде одного файла.
Правило именования файла: \texttt{name\_surname\_3.[py | ipnb]}. Например, если
вас зовут Иван Петров, то имя файла должно быть: \texttt{ivan\_petrov\_3.py} или \texttt{ivan\_petrov\_3.ipnb}.

\makebox[\linewidth]{\hrulefill}

\begin{figure}[h!]
  \centering
  \includegraphics[width=.8\linewidth]{images/boo}
\end{figure}

До Хэллоуина осталось всего полгода, самое время научиться отличать чудищ друг от друга. 
По ссылке \footnote{\url{https://gist.github.com/ktisha/c2d540df52be497c89ceaf27169b2bab}} находится датасет, содержащий информацию, которая поможет нам научиться отличать призраков, гоблинов и гулей друг от друга. Значения колонок указаны в заголовке файла, в качестве меток классов будет использоваться последняя колонка. 

\paragraph{1} Реализуйте класс \pythoninline{Node} для хранения узла в дереве принятия решений. Класс должен хранить ссылки на свои поддеревья в переменных \pythoninline{false_branch} и \pythoninline{true_branch}, а также предикат по которому происходит деление на поддеревья. Hint: предикат удобно хранить в виде номера признака, по которому происходит деление выборки, и его значения. 

\paragraph{2} В качестве критерия информативности в этой задаче мы будем использовать энтропийный критерий. Реализуйте функцию \pythoninline{entropy}, вычисляющую энтропию для некоторого подмножества объектов.

\paragraph{3} Реализуйте рекурсивный алгоритм построения дерева решения в виде класса \pythoninline{DecisionTree}. Структура класса приведена ниже: 

\begin{python3}
class DecisionTree:
    def build(self, X, y, score=entropy):
        # рекурсивный алгоритм построения дерева
        return self
        
    def predict(self, x):
        ... 
\end{python3}
\clearpage
Метод \pythoninline{build} должен:
\begin{itemize}
	\item Оценить информативность всех возможных предикатов для всех признаков с помощью функции \pythoninline{score}. 
	Для построения всех возможных предикатов для конкретного признака нужно определить уникальные значения данного признака. 
	Следующий шаг -- сконструировать пороговые условия для признака относительно полученных уникальных значений. 
	Обратите внимание, что признаки в датасете двух типов -- номинальные и количественные. 
	Для номинальных признаков количество предикатов будет равно количеству уникальных значений признака и пороговое условие превратится в проверку признака на равенство.
	\item Выбрать наилучшее с точки зрения информативности разбиение.
	\item Для наилучшего разбиения рекурсивно построить правое и левое поддеревья.
\end{itemize}

\paragraph{4} Реализуйте метод \pythoninline{predict}, принимающий объект \pythoninline{x} и возвращающий метку класса. 

\paragraph{5} Для визуализации понадобится библиотека pillow \footnote{\url{http://pillow.readthedocs.io/en/3.1.x/reference/ImageDraw.html}}. Реализуйте методы \pythoninline{getwidth} и \pythoninline{getdepth}. Дополните функцию \pythoninline{drawnode} для визуализации дерева.

\begin{python3}
def drawtree(tree, path='tree.jpg'):
    w = getwidth(tree) * 100
    h = getdepth(tree) * 100

    img = Image.new('RGB', (w, h), (255, 255, 255))
    draw = ImageDraw.Draw(img)

    drawnode(draw, tree, w / 2, 20)
    img.save(path, 'JPEG')

def drawnode(draw, tree, x, y):
    if isinstance(tree, Node):
        shift = 100
        width1 = getwidth(tree.false_branch) * shift
        width2 = getwidth(tree.true_branch) * shift
        left = x - (width1 + width2) / 2
        right = x + (width1 + width2) / 2

        # получите текстовое представление предиката для текущего узла
        predicate = ...
         
        draw.text((x - 20, y - 10), predicate, (0, 0, 0))
        draw.line((x, y, left + width1 / 2, y + shift), fill=(255, 0, 0))
        draw.line((x, y, right - width2 / 2, y + shift), fill=(255, 0, 0))
        drawnode(draw, tree.false_branch, left + width1 / 2, y + shift)
        drawnode(draw, tree.true_branch, right - width2 / 2, y + shift)        
    else:
        draw.text((x - 20, y), tree, (0, 0, 0))

\end{python3}

\paragraph{6} Какие предикаты влияют на классификацию объекта как класс "Goblin"?


\end{document}
