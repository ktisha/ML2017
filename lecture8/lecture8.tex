\documentclass[10pt]{beamer}

\usepackage[T2A]{fontenc}
\usepackage[utf8]{inputenc}
\usepackage[russian,english]{babel}
\usepackage{subfig}
\usefonttheme[onlymath]{serif}

\usetheme[progressbar=frametitle]{metropolis}
\usepackage{appendixnumberbeamer}

\usepackage{booktabs}
\usepackage[scale=2]{ccicons}

\usepackage{pgfplots}
\usepgfplotslibrary{dateplot}

\usepackage{xspace}
\newcommand{\themename}{\textbf{\textsc{metropolis}}\xspace}
\newcommand{\TODO}[1]{\textbf{\textcolor{red}{TODO: #1}}}

\date{}
\author{Екатерина Тузова}


\usepackage{gensymb}

\title{Лекция 8}
\subtitle{Нейронные сети}

\begin{document}	

\section{Разбор летучки}

\frame{\titlepage}

\begin{frame}\frametitle{Модель McCulloch-Pitts}
  $X = \mathbb{R}^n$, ${Y = \left\{ -1, + 1\right\}}$\\
	\pause
	$$a(x,w) = \sigma(\sum\limits_{j=1}^n w_j x^j - w_0) = \sigma(\langle \mathbf{w}, \mathbf{x} \rangle)$$\\
  $x^j$ — признаки объекта, $j = 1,\dots, n$ \\	
	$w_j \in R$ -- веса признаков\\
	$\sigma(s)$ -- функция активации (например, $\sign$)
	\pause
	\begin{figure}[htbp]
	  \includegraphics[height=100pt, keepaspectratio = true]{images/neuron-scheme}   
	\end{figure}
\end{frame}

\begin{frame}{Нейрон}
	\begin{figure}[htbp]
	  \includegraphics[height=200pt, keepaspectratio = true]{images/neuron}   
	\end{figure}
\end{frame}

\begin{frame}{Линейные алгоритмы классификации и регрессии}
	Задача классификации: \\
	$$a(\mathbf{x},\mathbf{w}) = \sign \langle \mathbf{w}, \mathbf{x} \rangle $$\\
	\bigbreak
	\bigbreak
	$$Q(w;X^l) = \sum\limits_{i=1}^l \mathcal{L} (\langle \mathbf{w}, \mathbf{x_i}\rangle y_i) \rightarrow \min\limits_w$$\\
\end{frame}

\section{Какой класс функций можно реализовать нейроном?}

\begin{frame}{Нейронная реализация логических функций}
	\begin{figure}[htbp]
	  \includegraphics[height=60pt, keepaspectratio = true]{images/OR}   
	\end{figure}
	
	\begin{figure}[htbp]
	  \includegraphics[height=60pt, keepaspectratio = true]{images/AND}   
	\end{figure}
\end{frame}

\begin{frame}{Нейронная реализация логических функций}
	Функции И, ИЛИ от бинарных переменных $x_1$ и $x_2$:\\
	$$x_1 \wedge x_2 = x_1 + x_2 - \frac{3}{2} > 0$$\\
	$$x_1 \vee x_2 = x_1 + x_2 - \frac{1}{2} > 0$$\\
\end{frame}

\begin{frame}{Логическая функция XOR}
	\begin{figure}[htbp]
	  \includegraphics[height=150pt, keepaspectratio = true]{images/XOR}   
	\end{figure}
\end{frame}

\begin{frame}{Логическая функция XOR}
	Функция $x_1 \oplus x_2 = [x_1 \neq x_2]$ не реализуема одним нейроном.\\
	\bigbreak
	\pause
	Два способа реализации:\\
	\begin{enumerate}[<+->]
		\item Добавлением нелинейного признака:\\
		$$x_1 \oplus x_2 = [x_1 + x_2 + 2 \textcolor{orange}{x_1x_2} - \frac{1}{2} > 0]$$
		\item Сетью (двухслойной суперпозицией) функций И, ИЛИ:\\
		$$x_1 \oplus x_2 = [(x_1 \vee x_2) - (x_1 \wedge x_2) - \frac{1}{2} > 0]$$
		\end{enumerate}
\end{frame}

\begin{frame}{Логическая функция XOR}
	\begin{figure}[htbp]
	  \includegraphics[height=150pt, keepaspectratio = true]{images/XOR1}   
	\end{figure}
\end{frame}

\begin{frame}{Любую ли функцию можно представить нейросетью?}
	\begin{itemize}[<+->]
		\item Двухслойная сеть в $\left\{0, 1 \right\}^n$ позволяет реализовать произвольную булеву функцию.
		\item Двухслойная сеть в $\mathbb{R}^n$ позволяет отделить произвольный выпуклый многогранник.
		\item Трёхслойная сеть $\mathbb{R}^n$ позволяет отделить произвольную многогранную область, не обязательно выпуклую.
		\item С помощью линейных операций и одной нелинейной функции активации $\sigma$ можно приблизить любую непрерывную функцию с любой желаемой точностью
	\end{itemize}
\end{frame}

%\begin{frame}{Практические рекомендации}
%	\begin{enumerate}[--]
%    	\item Двух-трёх слоёв обычно достаточно.
%	  \item Можно достраивать нейроны в произвольных местах сети по необходимости.
%	\end{enumerate}
%\end{frame}

\begin{frame}{Многослойная нейронная сеть}
	Пусть $Y = \mathbb{R}^M$, два слоя в сети.\\
	
	\begin{figure}[htbp]
	  \includegraphics[height=150pt, keepaspectratio = true]{images/neural_network}   
	\end{figure}
\end{frame}

\begin{frame}{Многослойная нейронная сеть}
	\begin{figure}[htbp]
	  \includegraphics[height=100pt, keepaspectratio = true]{images/neural_network1}   
	\end{figure}
\end{frame}

\begin{frame}{Стохастический градиент}
  ${Q}(\mathbf{w}) = \sum\limits_{i=1}^l \mathcal{L}(w, x_i, y_i) \rightarrow \min\limits_w$ \\
  \bigbreak
  \begin{algorithmic}[1]
    \Function{stochastic\_gradient}{$X^l$, $\alpha$, $\eta$}
     \State Перемешать данные в $X^l$
     \State Инициализировать $\mathbf{w}$
     \State ${Q}(\mathbf{w}) = \sum\limits_{i=1}^l \mathcal{L}(\langle \mathbf{w}, \mathbf{x_i} \rangle y_i)$
     \MRepeat [пока $Q$ и/или $w$ не стабилизируются] 
       \State Взять $x_i$ из $X^l$
       \State Потеря: $\varepsilon_i = \mathcal{L}(w, x_i, y_i)$
       \State Градиентный шаг: $w =  w - \alpha \triangledown \mathcal{L}(w, x_i, y_i)$
       \State Оценить $Q = (1-\eta)Q + \eta \varepsilon_i$
     \EndRepeat
    \EndFunction
  \end{algorithmic}  
\end{frame}

\section{Сколько операций потребуется для вычисления градиента в точке?}

\begin{frame}{Идея}
  \alert{Идея}: Сохранять некоторые промежуточные результаты в узлах сети.
\end{frame}

\begin{frame}{Задача дифференцирования суперпозиции функций}
	Выходные значения $a^m(x_i)$, $m = 1,\dots,M$ на объекте $x_i$:\\
	$$a^m(x_i) = \sigma_m (\sum\limits_{h=0}^H w_{hm} \textcolor{orange}{u^h(x_i)}), \qquad \textcolor{orange}{u^h(x_i)} = \sigma_h (\sum\limits_{j=0}^n w_{jh} x_i^j)$$\\
	\bigbreak
	\pause
	Возьмем $\mathcal{L}_i(w)$ -- средний квадрат ошибки:\\
	$$\mathcal{L}_i(w) = \frac{1}{2} \sum\limits_{m=1}^M (a^m(x_i) - y_i^m)^2$$\\
	\bigbreak
	\pause
	Промежуточная задача: найти частные производные\\
	$$\frac {\partial \mathcal{L}_i(w)}{\partial a^m} \hspace{5mm} \frac {\partial \mathcal{L}_i(w)}{\partial u^h}$$
\end{frame}

\begin{frame}{Быстрое вычисление градиента}
  Ошибка на выходном слое:	
	$$\frac{\partial \mathcal{L}_i(w)}{\partial a^m} = a^m (x_i) - y_i^m = \varepsilon^m_i$$ 
	\bigbreak
	\pause
	Ошибка на скрытом слое:	
	$$\frac{\partial \mathcal{L}_i(w)}{\partial u^h} = \sum\limits_{m=1}^M (a^m(x_i) - y_i^m) \sigma'_m w_{hm} = \sum\limits_{m=1}^M \varepsilon^m_i \sigma'_m w_{hm} = \varepsilon^h_i$$
	\bigbreak
	\pause
	$\varepsilon_i^h$ вычисляется по $\varepsilon_i^m$, если запустить сеть «задом наперёд»:\\
	
	\begin{figure}[htbp]
	  \includegraphics[height=50pt, keepaspectratio = true]{images/backpropagation}   
	\end{figure}

\end{frame}

\begin{frame}{Быстрое вычисление градиента}
	Теперь, имея частные производные $\mathcal{L}_i(w)$ по $a^m$ и $u^h$, легко выписать градиент $\mathcal{L}_i(w)$ по весам $w$.\\
	\bigbreak
	\pause
	\alert{Вопрос}: как?
\end{frame}

\begin{frame}{Быстрое вычисление градиента}
	$$\frac {\partial \mathcal{L}_i(w)}{\partial w_{hm}} = \frac {\partial \mathcal{L}_i(w)}{\partial a^m} \frac {\partial a^m}{\partial w_{hm}} = \varepsilon^m_i \sigma'_m u^h(x_i), \hspace{50mm}$$ $$m = 1,\dots,M, \hspace{2mm} h = 0,\dots, H$$\\
	\bigbreak
	\pause
	$$\frac {\partial \mathcal{L}_i(w)}{\partial w_{jh}} = \frac {\partial \mathcal{L}_i(w)}{\partial u^h} \frac {\partial u^h}{\partial w_{jh}} = \varepsilon_i^h \sigma'_h x_i^j, \hspace{50mm}$$ $$h = 1,\dots,H, \hspace{2mm} j = 0,\dots, n$$ \\
\end{frame}

{\foot{Backpropagation}
\begin{frame}{Алгоритм обратного распространения ошибки}
  \begin{algorithmic}[1]
    \Function{Backpropagation}{$X^l$, $H$, $\alpha$, $\eta$}
     \State \dots
     \MRepeat [пока $Q$ не стабилизируются] 
       \State Взять $x_i$ из $X^l$
       \pause
       \State 	$\begin{cases}
	\hspace{5mm} u^h_i = \sigma_h (\sum\limits_{j=0}^J w_{jh} x_i^j), \hspace{4mm} h = 1, \dots, H\\
	\hspace{5mm} a^m_i = \sigma_m (\sum\limits_{h=0}^H w_{hm} u_i^h), \hspace{4mm} \varepsilon_i^m = a_i^m - y_i^m, \hspace{4mm} m = 1, \dots, M\\
	\hspace{5mm} \mathcal{L}_i = \frac{1}{2} \sum\limits_{m=1}^M (\varepsilon_i^m)^2\\
	\end{cases}$\\
	\pause
	$\begin{cases} \hspace{5mm} \varepsilon_i^h = \sum\limits_{m=1}^M \varepsilon_i^m \sigma'_m w_{hm}, \hspace{4mm} h = 1, \dots, H\end{cases}$\\
	\pause
	$\begin{cases} \hspace{5mm} w_{hm} = w_{hm} - \alpha \varepsilon_i^m \sigma'_m u_i^h, \hspace{4mm} h = 0, \dots, H, \hspace{4mm} m = 1, \dots, M\\
	\hspace{5mm} w_{jh} = w_{jh} - \alpha \varepsilon_i^h \sigma'_h x_i^j, \hspace{4mm} j = 0, \dots, n, \hspace{4mm} h = 1, \dots, H\\
	\hspace{5mm} Q = (1 - \eta)Q + \eta \mathcal{L}_i \end{cases}$
     \EndRepeat
    \EndFunction
  \end{algorithmic}    
\end{frame}
}

\begin{frame}{Особенности}
	\begin{enumerate}[<+->]
	\item[+] Эффективность: градиент вычисляется за время, сравнимое со временем вычисления самой сети
	\item[+] Легко обобщается на любые $\sigma$, $\mathcal{L}$
	\item[+] Возможно динамическое (потоковое) обучение
	\item[+] На сверхбольших выборках не обязательно брать все $x_i$
	\item[+] Возможность распараллеливания
	\bigbreak
	\item[--] Возможна медленная сходимость
	\item[--] Застревание в локальных минимумах
	\item[--] Проблема переобучения
	\item[--] Подбор комплекса эвристик
	\end{enumerate}
\end{frame}


\begin{frame}{Стандартные эвристики для метода SG}
	Применимы все те же эвристики, что и в обычном SG.\\
	\bigbreak
	\pause
	Напомните.
\end{frame}

\begin{frame}{Стандартные эвристики для метода SG}
	\begin{itemize}[<+->]
		\item Инициализация весов
		\item Порядок предъявления объектов
		\item Оптимизация величины градиентного шага
		\item Регуляризация (сокращение весов)
	\end{itemize}
\end{frame}

\begin{frame}{Новые проблемы}
	\begin{itemize}[<+->]
		\item Выбор функций активации в каждом нейроне
		\item Выбор числа слоёв и числа нейронов
	\end{itemize}
\end{frame}

\begin{frame}{Примеры функций активации}
	\begin{itemize}[<+->]
		\item Сигмоида
		$$\sigma(x) = \frac{1}{1 + e^{-x}}$$
		\item Гиперболический тангенс
		$$\tanh(x) = 2\sigma(2x) - 1$$
		\item Rectified Linear Unit
		$$f(x) = \max(0, x)$$
	\end{itemize}
\end{frame}

%\begin{frame}{Ускорение сходимости}
%	\begin{itemize}[<+->]
%		\item Тщательный подбор начального приближения
%		\item Адаптивный градиентный шаг
%		\item Метод сопряжённых градиентов и chunking -- разбиение суммы $Q(w) = \sum\limits_{i=1}^l \mathcal{L}_i(w)$ на блоки
%	\end{itemize}
%\end{frame}

%\begin{frame}{Начальное приближение}
%	Нейроны настраиваются как отдельные линейные алгоритмы:
%	\begin{itemize}
%		\item либо по случайной подвыборке $X' \subseteq X^l$
%		\item либо по случайному подмножеству входов
%		\item либо из различных случайных начальных приближений
%	\end{itemize}
%	Tем самым обеспечивается различность нейронов.
%\end{frame}

%\begin{frame}{Оптимизация структуры сети}
%	Динамическое наращивание сети.\\
%	\begin{itemize}[<+->]
%		\item Обучение при заведомо недостаточном числе нейронов $H$
%		\item После стабилизации  $Q(w)$ -- добавление нового нейрона и его инициализация путём обучения
%		\item Снова итерации BackProp
%	\end{itemize}
%	\vspace{5mm}
%	Противоположная идея -- прореживание сети.
%\end{frame}

\section{Практическое применение}

\begin{frame}{Специальные нейронные сети}
	Почему их снова стали использовать?\\
	\pause
	\begin{enumerate}[--]
		\item Большие вычислительные возможности (в том числе
		распределенные вычисления)
		\item Большие объемы данных
		\item Новые методы предобучения, новые архитектуры
	\end{enumerate}
\end{frame}

\section{Свёрточные сети}

{\foot{http://www.cs.nyu.edu/~yann/talks/lecun-ranzato-icml2013.pdf}
\begin{frame}{Свёрточные сети}
  \begin{enumerate}
    \item Пиксель изображения сильнее связан с соседними пикселями (локальная корреляция)
		\item Объект может встретиться в любой части изображения (инвариантность к перемещению)
  \end{enumerate}
  \pause
  \begin{enumerate}
    \item Каждый из нейронов подсоединен только к небольшой окрестности изображения
    \item Нейроны обладают одними и теми же весами 
  \end{enumerate}
\end{frame}
}

\begin{frame}{Локально-связанный слой}
	\begin{figure}[htbp]
	  \includegraphics[height=200pt, keepaspectratio = true]{images/local_layer}   
	\end{figure}
\end{frame}

\begin{frame}{Операция свертки}
	\begin{figure}[htbp]
	  \includegraphics[height=200pt, keepaspectratio = true]{images/conv}   
	\end{figure}
\end{frame}

\begin{frame}{Простейшая свертка}
	\begin{figure}[htbp]
	  \includegraphics[height=150pt, width=\textwidth, keepaspectratio = true]{images/conv0}   
	\end{figure}
\end{frame}

\begin{frame}{Простейшая свертка}
	\begin{figure}[htbp]
	  \includegraphics[height=150pt, width=\textwidth, keepaspectratio = true]{images/conv2}   
	\end{figure}
\end{frame}

\begin{frame}{Простейшая свертка}
	\begin{figure}[htbp]
	  \includegraphics[height=150pt, width=\textwidth, keepaspectratio = true]{images/conv1}   
	\end{figure}
\end{frame}

\begin{frame}{Простейшая свертка}
	\begin{figure}[htbp]
	  \includegraphics[height=150pt, width=\textwidth, keepaspectratio = true]{images/conv3}   
	\end{figure}
\end{frame}

{\foot{Convolutional neural net}
\begin{frame}{Свёрточная сеть}
  Извлечение признаков во время обучения.\\

	\begin{figure}[htbp]
	  \includegraphics[height=150pt, keepaspectratio = true]{images/deep_learning}   
	\end{figure}
\end{frame}
}

{\foot{https://www.cs.toronto.edu/~fritz/absps/imagenet.pdf}
\begin{frame}{ImageNet}
  \centering
  \includegraphics[width=0.9 \textwidth, keepaspectratio = true]{images/imagenet}   
\end{frame}
}

\begin{frame}{ImageNet}
  \centering
  \includegraphics[width=0.9 \textwidth, keepaspectratio = true]{images/top5}   
\end{frame}

\begin{frame}{Похожие изображения}
  \centering
  \includegraphics[width=0.9 \textwidth, keepaspectratio = true]{images/similar}   
\end{frame}

\begin{frame}{Похожие изображения}
  \centering
  \includegraphics[width=\textwidth, keepaspectratio = true]{images/similar2}   
\end{frame}

{\foot{https://arxiv.org/pdf/1508.06576v1.pdf}
\begin{frame}{Стилизация изображений}
  \centering
  \includegraphics[width=\textwidth, keepaspectratio = true]{images/style}   
\end{frame}
}

\section{Рекуррентные нейронные сети}

\begin{frame}{RNN}
  \centering
  \includegraphics[width=0.9 \textwidth, keepaspectratio = true]{images/RNN}   
\end{frame}

\begin{frame}{Пример}
  Текст состоит из последовательности слов. Хотелось бы подавать слова на вход нейросети по очереди, но так, чтобы сеть помнила контекст.
\end{frame}

{\foot{http://karpathy.github.io/2015/05/21/rnn-effectiveness/}
\begin{frame}{Шекспир}
  \centering
  \includegraphics[width=0.9 \textwidth, keepaspectratio = true]{images/shakespeare}   
\end{frame}
}

{\foot{http://karpathy.github.io/2015/05/21/rnn-effectiveness/}
\begin{frame}{Генерация кода}
  \centering 
  \includegraphics[width=0.9 \textwidth, keepaspectratio = true]{images/code}   
\end{frame}
}

\begin{frame}{Автоматический перевод}
  \centering 
  \includegraphics[width=0.9 \textwidth, keepaspectratio = true]{images/translate}   
\end{frame}

\section{RNN + CNN}

{\foot{http://www.cs.toronto.edu/~rkiros/adv\_L.html}
\begin{frame}{Подпись к изображениям}
  \centering 
  \includegraphics[width=0.9 \textwidth, keepaspectratio = true]{images/subscr}   
\end{frame}
}

{\foot{http://www.cs.toronto.edu/~rkiros/adv\_L.html}
\begin{frame}{Подпись к изображениям}
  \centering 
  \includegraphics[width=0.9 \textwidth, keepaspectratio = true]{images/image_gen}   
\end{frame}
}  

\begin{frame}[standout]
  Вопросы?
\end{frame}

\appendix

\begin{frame}\frametitle{Что почитать по этой лекции}
  \begin{itemize}
    \item \href{http://cs231n.github.io/neural-networks-1/}{Andrej Karpathy lecture notes (cs231n)}
    \item Tom Mitchell "Machine Learning" Chapter 4.5
  \end{itemize}
\end{frame}

\begin{frame}{На следующей лекции}
	\begin{itemize}
	  	\item[--] Задача максимизации зазора - аналог классификатора с регуляризацией
    \item[--] Двойственная задача 
    \item[--] Что такое опорный вектор
    \item[--] Регуляризация
    \item[--] Решение для неразделимых выборок
    \item[--] Kernel trick    
	\end{itemize}
\end{frame}

\end{document}
